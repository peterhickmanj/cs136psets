 \documentclass[11pt]{article}
%\usepackage[margin=1.25in]{geometry}
\usepackage{hyperref}
\usepackage{ifthen}
\usepackage{enumerate,fullpage}
\usepackage{amssymb}
\usepackage{amsmath}
\usepackage{graphicx}
%\usepackage{bbm}
\usepackage{version}
\usepackage{pdfsync}
\usepackage{color}

\pagestyle{plain}

\newcommand{\points}[1]{\textbf{[#1 Points]}}

\def\B{{\mathbb B}}
\def\C{{\mathbb C}}
\def\D{{\mathbb D}}
\def\H{{\mathbb H}}
\def\M{{\mathbb M}}
\def\N{{\mathbb N}}
\def\0{{\mathbb 0}}
\def\P{{{\mathbb P}}}
\def\Q{{\mathbb Q}}
\def\R{{\mathbb R}}
\def\T{{\mathbb T}}
\def\Z{{\mathbb Z}}

\newcommand{\bmat}{\begin{bmatrix}}
\newcommand{\emat}{\end{bmatrix}}
\newcommand{\bpmat}{\begin{pmatrix}}
\newcommand{\epmat}{\end{pmatrix}}
\newcommand{\bvmat}{\begin{vmatrix}}
\newcommand{\evmat}{\end{vmatrix}}
\newcommand{\qed}{\quad\square}
\newcommand{\sol}{\smallskip\textbf{Solution: }}
\newcommand{\Tr}{\text{Tr}}
\renewcommand{\ker}{\text{ker}}
\newcommand{\im}{\text{im}}
\newcommand{\rank}{\text{rank}}
\newcommand{\nullity}{\text{nullity}}
\renewcommand{\dim}{\text{dim}}
\renewcommand{\th}{\text{th}}


\begin{document}

\title{CS 136 Assignment 6: Sponsored Search Auctions}
	\author{David C. Parkes  \\
	School of Engineering and Applied Sciences, Harvard University \\
	Out: Sunday, March 13 2016\\
	Due: {\bf 5pm} sharp: {\bf Friday, March 25 2016}\\
	Submissions to {\tt Canvas}}
\date{}

\maketitle

\noindent \points{Total: 135} {\em Submit a single zip file containing
  your code and your writeup.} The total number of points includes a
small bonus for good performance in an in-class tournament.

%{\bf College}:
This is a group-assignment to be completed by groups of
up to 2 students. While you are permitted to discuss your agent
designs with other students, each group must develop its own code and
write-up its own submission. (If you need help finding a partner
please post to Piazza.)

Your submissions should be made to Canvas: the code must be in .py files and the writeup of the
analysis must be in PDF format. Both partners must contribute to all aspects of the assignment.
It should go without saying that you must write all of the code yourself, per Harvard's Academic
Integrity Policy. If you use any code whatsoever from any source other than the official Python
documentation, you \textbf{must} acknowledge the source in your writeup.

\section{Introduction}

You will program bidding agents to participate in a generalized
second-price (GSP) auction.  The first agent you will design uses a
{\em balanced bidding} strategy, which is inspired by Defn.~10.5 in
the reading.  You will also design an agent to do as well as possible
in a competitive environment, and in the presence of daily budget
constraints on how much it can spend (see, e.g. the discussion in
Section~10.6.1 in the reading).  You will also explore the effect that
the design of payment rules has on revenue.


\section{Setup}
	\begin{description}

		\item[Generating .py-files:] Pick a group name, perhaps based on your initials, so it will be unique in the class. Run \verb+python start.py GROUPNAE+, substituting your group name for \verb+GROUPNAME+. This will create appropriately named template files for your clients. For example, if your group name was ``abxy'':
			\begin{verbatim}> python start.py abxy
				Copying bbagent_template.py to abxybb.py...
				Copying bbagent_template.py to abxybudget.py...
				All done.  Code away!
			\end{verbatim}
			In each of these newly generated files, you will need to change the class name \verb+BBAgent+  to your teamname plus client specification (e.g. for the balanced bidding agent: \verb+class Abxybb+).
	
                      \item[The simulator:] You are given an
                        ad-auction simulator.  We ignore quality
                        effects, so that in a time period, only the
                        position matters in determining the
                        probability of a click and thus the expected
                        number of clicks received by an ad.

The simulator works as follows:
%
			\begin{itemize}
				\item Time proceeds in discrete periods ($0, 1, 2, \ldots, 47)$. Each period simulates 30 minutes and includes multiple auctions,
and the 48 periods correspond to a day. The simulator can be used to simulate
multiple days by increasing the number of iterations.

				\item In the first period, an agent's
                                  bid is queried through
                                  \verb+initial_bid()+ whereas for all
                                  subsequent periods, the bid is
                                  queried through a call to
                                  \verb+bid()+. The simulator tracks
                                  money and values in integer numbers
                                  of cents.
The same bid value is used for every auction that
occurs within a period.
To encourage competition, the number of available positions is one less than
the number of agents in the market.

\item Each period is simulated by collecting bids, assigning positions,
  determining the expected number of clicks received by each bidder,
  and determining payments and utilities.  Sometimes there is a {\em
    reserve price} (see Section~4). The positions are assigned
in order of submitted bid.
%
\item Number of clicks: Let $c_j^t$ denote the number of clicks
  received by an ad in position $j$ (from 1 to the number of
  positions) in period $t$. For the top position, this
  follows a cosine curve:
%
\begin{align}
 c_1^t &= \mbox{round}(30 \cos(\frac{\pi t}{24}) + 50),
\quad t = 0, 1, \ldots,
\end{align}
%
where $\mbox{round}(x)$ returns the nearest integer value to $x$.
The number of clicks starts at 80, falls to 20 by period 24,
and  increases  to 80 by the end of a day.

The number of clicks decreases according to a multiplicative
position effect. The number of clicks received by an ad in
position $j$ ($j>1$) in period $t$ is
%
\begin{align}
c_j^t &= 0.75^{(j-1)}c_1^t,
\end{align}
%
so that it decreases by a factor of 75\% from  position to position.
%
\item The price in the GSP auction depends on the next highest bid.
Given this, the utility in period $t$ to
agent $i$ occupying position $j$
is

\begin{align}
u_i^t &=  c^t_j(v_i - \mathit{pay}_{\mathrm{gsp},j}^t) = c^t_j(v_i - b^t_{j+1}),
\end{align}
%
where $v_i$ is the per-click value,
and $\mathit{pay}^t_{\mathrm{gsp},j}$ the
price for position $j$ under the GSP auction and equals
$b^t_{j+1}$, which
is the amount of the next highest
bid (or zero
or the reserve if there
is no such bid.)
Thus is the number of clicks received multiplied by the
utility (= value - price) per click.
%
%				
\item Budget constraint: The total payment by an agent in position $j$
  in period $t$ is $c_j^t b^t_{j+1}$. Each agent has a {\em daily
    budget constraint}, but for most of the assignment this will be
  high enough not to matter. For the competition this is \$600/day and
  it will matter.
 An agent  with budget still available at the start
of a period is still allowed to bid. For this reason, an
agent  is allowed to slightly over-spend its budget.
Once the budget is exhausted the
bid in all
subsequent periods  of the day
must be \$0 (the
  simulator will ensure this if you try to bid more than \$0.)
%				
\item At the start of each day, the per-click value $v_i$ of an agent
  is sampled independently from the uniform distribution on
  $[0.25,1.75]$.  The simulator also runs additional permutations of
  value assignments to agents to improve the statistical
  significance of the estimated utility.
If the number of agents is at most 5 then
all permutations
 of values to agents
are tested. Otherwise,
  120 random permutations are tested. (You can change this threshold
  with the \verb+--perms+ option)
%
\item The score of an agent is the total utility over all 48 periods,
  and averaged over all permutations of values to agents that are
  used.  In addition, multiple iterations can be run. This has the
  effect of repeating the test over multiple days.  In this case the
  score is the average total utility over all permutations and all
  days.  For a single day, the total utility to agent $i$ with value
  $v_i$ is $\sum_{t=0}^{47} u_i^t = \sum_{t=0}^{47} c_{x_{it}}^t(v_i -
  b_{x_{it}+1}^t)$, where $x_{it}$ is the position assigned in period $t$ to agent $i$ (and
  $c_{x_{it}}^t=0$ if the agent is not assigned to a position in period $t$.
{\em Any unspent  budget has no effect on utility or score}.
			\end{itemize}
%
		\item[Source code:] Familiarize yourself with the provided code. You will need to change the agent created by \verb+start.py+, as well as the file \verb+vcg.py+. You'll want to take a look at the simple truthful-bidding agent in \verb+truthful.py+, as well as the the GSP implementation in \verb+gsp.py+.

		\item[Testing:] Here are some initial test commands to run. The following command gives a list of helpful command line parameters:
			\begin{verbatim}
				> python auction.py --h
			\end{verbatim}

			The following command can be used to test the code. It runs the auction with five truthful agents for two periods (rather than the default of 48 periods). The \verb+--perms 1+ command forces the simulator to assign only one permutation of value to the agents. %\textbf{SS: be more precise what this command does, what does Truthful,5 do, for example?}:
			\begin{verbatim}
				> python auction.py --loglevel debug --num-rounds 2 --perms 1 Truthful,5
			\end{verbatim}

The number of periods defaults to 48 if this is not set.

The following command runs the auction with a reserve price of 40
cents. The \verb+--iters 2+ command specifies that the 48 periods will
be repeated twice, with different value samples (i.e., two days.)
%
The
%
\verb+--seed INT+ (where INT is any integer) command
 sets the seed of the random number generators in the simulator and
allows for repeatability.
%
			\begin{verbatim}
				> python auction.py  --perms 1 --iters 2 --reserve=40 --seed 1  Truthful,5
			\end{verbatim}


		\item[Tips]
			\begin{itemize}
				\item {\em Permutations}: If you're
running an experiment on a population of agents that each have the
same strategy,
use \verb+--perms 1+ to make the code run faster. In this case this is
probably OK because you're likely comparing the average utility or
revenue of the auction, and not looking at specific agents.
%
				\item {\em Pseudo-random numbers}: If you're trying to track down a bug, or understand what's going on with some specific case, use \verb+--seed INT+ to fix the random seed and get repeatable value distributions and tie breaking.
			\end{itemize}

\item[Comments]
\begin{itemize}
\item {\em Numbering convention}: The positions (slots) in this
  write-up are 1-indexed (top slot is slot number 1) but 0-indexed in
  the code. Don't be confused by this!
%
\item {\em Ties}: In the case of two tied bids one of the two is randomly chosen to be allocated th higher slot.
%
\item
{\em State}: The agent you write can access history from the previous period within each day.
This could allow for more sophisticated strategies.

\item {\em Workarounds}: The code is designed so that it's hard to mess up the main simulation accidentally, but because everything is in the same process, it is still possible to cheat by directly modifying the simulation data structures and such. Don't!
%
		\item {\em Bugs}: If you find bugs in the code,
please let us know. If you want to improve the logging or stats or performance or add animations or graphs, feel free :) Send those changes along too.
%
    		\item {\em Questions}:
If something is unclear about the assignment, please ask a question on Piazza. Please don't post solution code but it is OK
to post code snippets are fine (use your judgment).
    	\end{itemize}
	\end{description}
	

\section{The Balanced Bidding Agent}

	\begin{enumerate}

		\item\points{30} {\bf Designing a bidding agent}

You are given a truthful bidding agent in \verb+truthful.py+.

In
\verb+GROUPNAMEbb.py+, write an agent that best-responds to the
bids of agents in the previous period. In particular, the agent follows
a {\em balanced bidding} strategy.  Consider period $t$.
Let $b_{-i}^{t-1}$ denote the bids from the
agents other than $i$ in the  period $t-1$.
Suppose there are $m$ positions. Balanced-bidding
proceeds as follows:
%
\begin{itemize}
\item Given bids $b_{-i}^{t-1}$, agent $i$
targets the position $j^\ast$ that maximizes
%
\begin{align}
\max_{j\in\{1,\ldots,m\}} [p_j\cdot (v_i - t_j)],
\end{align}
%
where $p_j$ is the position effect (this falls by a multiplicative
factor of $0.75$ each slot) and $t_j$ is the price the agent would pay
for position $j$ given bids $b^{t-1}_{-i}$. For example, in GSP auction, $t_j$
equals the $j$-th highest bid in $b_{-i}^{t-1}$.

%
\item ({\em Not expecting to win})
If price $t_{j^\ast}\geq v_i$ in this target position,
then bid $b^t_i=v_i$ in period $t$.
% [In particular, the agent
%is not expecting to win a position.]
%
		\item Otherwise:
%
\begin{itemize}
\item[(a)] ({\em Not going for the top}) If target position $j^\ast>1$,
then set  bid $b^t_i$ to satisfy
%
\begin{align}
p_{j^\ast}(v_i - t_{j^\ast}) = p_{j^\ast-1}(v_i - b^t_i).
\end{align}
%
%
\item[(b)] ({\em Going for the top}) If $j^\ast=1$,
then bid
$b^t_i=v_i$.
\end{itemize}
	\end{itemize}

        This strategy is motivated by the balanced bidding discussion
        the reading (Section~10.4.2).\footnote{Cary et al.  ``Greedy
          bidding strategies for keyword auctions" {\em Proc.  ACM EC
            2007} pp 262--271, show that this balanced bidding
          strategy converges to the spiteful Nash equilibrium of the
          GSP auction.}
%
	The file \verb+TEAMNAMEbb.py+ provides a skeleton of a
        balanced bidding agent. You will need to complete
the code to compute the  expected
        utility for each position and the optimal bid.

		\item \points{20} {\bf Experimental Analysis}

{\bf 			To answer the following questions, run the
			simulation with 5 agents. By default the
			budget is \$5000, which is
not binding. Leave it
			this way!}
		
			\begin{enumerate}			
				\item \points{10} What is the average utility of a population of truthful agents? What is the average utility of a population of balanced bidding agents? {\em  Compare the two cases and explain your findings.}


				Make use of the \verb+--perms+, \verb+--seed+, and \verb+--iters+ commands, e.g. \verb+--perms 1 --seed 2+ \verb+--iters 200+ would be a good starting point.

				\item \points{10} In addition, what is the
average utility of one balanced-bidding agent against 4 truthful agents, and
one truthful agent against 4 balanced-bidding agents?
For the new experiment,
make use of the \verb+--seed+, and \verb+--iters+ commands, but you will now
want to run multiple permutations. Note that you can add multiple agents types using:
			\begin{verbatim}
				> python auction.py  --perms 10 --iters 200 Truthful,4 abxybb,1
			\end{verbatim}
%

{\em What does this suggest about the incentives to follow the truthful vs. the balanced bidding strategy?}
			\end{enumerate}

\end{enumerate}

\section{Experiments with Revenue: GSP vs VCG auctions}

In this section we compare the revenue properties of the GSP and VCG
auctions with different reserve prices.
%
A reserve price, $r>0$, sets a minimum price for any position.  Used
carefully, reserve prices can increase revenue.
%
The reserve price in the GSP works as follows: only bids (weakly)
above $r$ can be allocated.  The agent in the lowest allocated
position pays the maximum of the reserve price and the maximum bid of
unallocated bidders. The VCG auction works in a similar way and is
explained below.\footnote{One way to think about it is that the
  reserve is made operational by including a ``dummy bidder'' whose
  bid is the amount of the reserve, and ignoring all bids with value
  less than $r$.}

\begin{enumerate}
		\item[3.] \points{55} {\bf Auction Design and Reserve Prices}

{\bf Run all simulations with 5 agents. Leave the budget to its default
of \$5000.}

			\begin{enumerate}
\item \points{20} Complete the code that runs
the VCG auction in \verb+vcg.py+. The
allocation rule is already implemented. You need to implement the
payment rule. Because of the reserve price, it is most convenient to
use the recursive form
of the VCG payment rule (see the proof of Theorem~10.3 in the
reading).

In particular, suppose there are 3 bidders with bids weakly greater
than the reserve price, and $b_1\geq b_2\geq b_3$, and say that $b_4$
is the fourth highest bid. Bidders 1--3 are allocated the top 3
positions.

Let $t_{\mathrm{vcg},i}(b)$ denote the expected payment by bidder $i$
in a single auction given bid profile $b$.  For bidder 3, this is
$t_{\mathrm{vcg},3}(b)=p_3 \max(r,b_{4})$. For bidders 1 and 2, in
positions 1 and 2 respectively, this is
$t_{\mathrm{vcg},i}(b)=(p_i-p_{i+1})b_{i+1} +t_{\mathrm{vcg},i+1}(b)$.

				\item \points{10} What is the auctioneer's revenue under GSP with no reserve price when all the agents use the balanced-bidding strategy? What happens as the reserve price increases? What is the revenue-optimal reserve price?

You can set the reserve price in the simulation with the command line argument \verb+--reserve INT+ (where INT is the reserve price in cents). Also use \verb+--perms 1+ and \verb+--iters 200+.
	

				
\item \points{10} What is the auctioneer's revenue under VCG with no
reserve price when all agents are truthful? What happens as the
reserve price increases? Explain your findings and compare with the
results of part (a).

				Again use the \verb+--perms+, \verb+--seed+, and \verb+--iters+ commands, e.g. \verb+--perms 1 --seed 2+ \verb+--iters 200+.
				
				
\item \points{10} Fix the reserve price to zero.
Explore what might happen if a search engine
switched over from the GSP to VCG design.
For this,
run the balanced-bidding agents in
GSP, and at period 24, switch to VCG, by using the \verb+--mech=switch+ parameter. What happens to the revenue?
%

				Again use the \verb+--perms+, \verb+--seed+, and \verb+--iters+ commands, e.g. \verb+--perms 1 --seed 2+ \verb+--iters 200+.
			
                              \item \points{5} In one paragraph,
state what
                                you learned from these exercises
                                about agent design,
                                auction design, and revenue? (There
                                is no specific right answer).
			\end{enumerate}

%\newpage
\end{enumerate}

\section{The Competition}
	
\begin{enumerate}
		\item[4.] \points{30} {\bf Budget constraints}

                  The balanced-bidding agent does not consider the
                  budget constraint when deciding how to bid.  In the
                  final part of the assignment, your task is to design
                  a {\em budget-aware agent}.

                  This agent will compete in the simulated GSP auction
                  against the agents submitted by other groups. You
                  might like to test your design against a variety of
                  other strategies.

                  For example, you could write an agent that measures
                  competition, or
tries to drive up the payments of other
                  agents so that they pay more and exhaust their
budgets! Example ideas include:
%
\begin{itemize}
\item Trying not to bid too much when
click-through rates are low.
%
\item Try to bid when other agents are not bidding very much
and the price is lower.
%
\end{itemize}

{\bf For the purpose of the competition, the daily budget constraint
  will be set to \$600 (use the \texttt{--budget} flag). We will run a
  GSP auction with a small reserve price.  Auctions will contain
  around 5 agents.}

The competition will be structured as a tournament, with agents placed
into groups of 5 or 6 and the top few agents making it into a
semi-final and final. The precise structure of the tournament will
depend on the number of agents submitted.

		\begin{enumerate}
			\item \points{25} In \verb+TEAMNAMEbudget.py+, write your  competition agent.

Describe in a few sentences how it works, why
it is designed this way,
and how you expect it to perform in the class competition.
		 {\em You are not expected to spend many hours on writing an optimal  agent, unless you want to. Consider a few possible strategies, try them out, pick the best one.}

			\item \points{5} Win the competition (!)
		Likely parameters for the competition are
		
		\verb+./auction.py --num-rounds 48 --mech=gsp --reserve=10 --iters 200+

(followed by the list of submitted agents)

		\end{enumerate}			
	\end{enumerate}

\end{document}
